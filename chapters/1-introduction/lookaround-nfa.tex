
\begin{center}
    \begin{tikzpicture}
        \node[draw, circle] (nodeb) at (0, 0) {};
        \node[draw, inner sep=0.5, align=center] (nodea) at (3.5, 0) {
            \begin{tikzpicture}
                \node[draw, circle, inner sep=3.33pt] (root) at (0, 0) {};
                \draw[->] ([xshift=-16pt]root.west) -- (root);
                \draw[->] (root) -- node[pos=0.2, inner sep=3.33pt, above] (a) {$a$} ([xshift=16pt]root.east);
                \node (space) at (0, -0.4) {};
            \end{tikzpicture} \\

            \begin{tabular}{c|c|c|c|c|c}
                \hline
                Index & 0 & 1 & 2 & 3 & 4 \\ \hline
                Succeeds & \cmark & \xmark & \cmark & \xmark & \xmark \\
            \end{tabular}
        };

        \draw[->] ([xshift=-16pt]nodeb.west) -- (nodeb);
        \draw[->] (nodeb) -- node[pos=0.2, above] (b) {$b$} (nodea);
        \draw[->] (nodea) -- ([xshift=16pt]nodea.east);
    \end{tikzpicture} \\ \vspace{0.3ex}
    \textit{NFA construction for \texttt{b(?=a)} against string \texttt{abac}} \\
    \textit{The Lookaround succeeds if the character at that index is an }\texttt{a}
\end{center}
